\documentclass[12pt, a4paper]{article}
\usepackage{url,graphicx,tabularx,array,geometry}
\usepackage[utf8]{inputenc}
\usepackage{paralist}
\usepackage{latexsym}
\usepackage{fancyhdr}
\usepackage{ textcomp }
\usepackage{ mathrsfs }
\usepackage{cancel}

\pagestyle{fancy}

\usepackage{amsmath}
\usepackage{amsfonts}
\usepackage{amssymb}

\DeclareUnicodeCharacter{00A0}{ }

\setlength{\parskip}{1ex} %--skip lines between paragraphs
\setlength{\parindent}{0pt} %--don't indent paragraphs

%-- Commands for header
\newcommand{\bs}{\ensuremath{\backslash}}
\renewcommand{\title}[1]{\textbf{#1}\\}
\renewcommand{\line}{\begin{tabularx}{\textwidth}{X>{\raggedleft}X}\hline\\\end{tabularx}\\[-0.5cm]}
\newcommand{\leftright}[2]{\begin{tabularx}{\textwidth}{X>{\raggedleft}X}#1%
& #2\\\end{tabularx}\\[-0.5cm]}
%\linespread{2} %-- Uncomment for Double Space
\begin{document}
\renewcommand{\headrulewidth}{0pt}
\fancyhf{}
\fancyhead[L]{
\leftright{\textbf{Zusammenfassung}}{Daniel Schmidt}
\line
\leftright{\textbf{Datenbanktheorie SS 16}}{}}
\fancyfoot[C]{\thepage}

\section*{Deduktive Datenbanksysteme}

\textbf{Problem:} Transitiver Abschluss ist in PL1 nicht formulierbar (mit zustandsabhängiger Formulierung möglich)\\
\textbf{Diskussion:} \\
Typ 5: $q_1(...), ..., q_n(...) :- .$\\
Typ 6: $q_1(...),...,q_n(...) :- p_1(...),...,p_m(...).$\\
$\Rightarrow$ Übungsaufgabe

Nur Typ 1 und Typ 4: $q(...) :- p_1(...),...,p_n(...), n \ge 0$ (ist die Hornklauselform und wird bei definiten Datenbanken genutzt) 

\subsection*{Definite Datenbanken}
\begin{equation}
\begin{split}
q(...) &:- . \text{(Fakt)}\\
q(...) &:- p_1(...),...,p_n(...). \text{(deduktive Regel, } p_{1-n} \text{ Teilziele)}
\end{split}
\end{equation}

\begin{itemize}
\item Mit IBen (Integritätsbedingungen) (+ Typ2, Typ3) ($:- p_1(...), ...,p_n(...)$)
\item Typ5 + Typ6 $\Rightarrow$ \textbf{Disjunktive Datenbank}
\item Definite Datenbank + negative Atome im Rumpf von Hornklauseln erlaubt $\Rightarrow$ Volles Datalog
\end{itemize}

\subsubsection*{Formulierung von Anfragen}
Klauseln vom Typ: $:- p_1(...),....,p_n(...)$, geschrieben $? - p_1(...),....,p_n(...)$

Beispiele:
\begin{itemize}
\item $? - ag(X,m).$
\begin{itemize}
\item Bedeutung: Welche Kurse bietet 'm' an?
\item DRC: (x) / ANGEBOT(X, m) 
\end{itemize}
\item $? - ag(a3,m).$
\begin{itemize}
\item Bedeutung: Bietet 'm' den Kurs 'a3' an?
\item DRC: () / ANGEBOT(a3, m) 
\end{itemize}
\item $? - ag(X,m), bl(X,s,j).$
\begin{itemize}
\item Bedeutung: Gib alle von 'm' angebotene Kurse, die 's' als Wiederholer belegt hat
\item DRC: (x) / ANGEBOT(x, 'm') $\wedge$ BELEGUNG(x, 's', 'y') 
\end{itemize}
\item Wie ist (x) / ANGEBOT(x, 'm') $\wedge (\exists $y) BELEGUNG(x, 's', y) formulierbar?
\begin{itemize}
\item Bedeutung: Gib die Dozenten der von s als Wiederholer belegte Kurse
\item Formulierung: $? - Ksm(X). \\ Ksm(X) :- ag(X, m), bl(X,s,y)$
\item Bequemer: $?- ag(X, Y*), bl(X,s,y).$, * Kennzeichnet die Ausgabevariable
\end{itemize}
\end{itemize}

In Anfragesprachen werden Vergleichsausdrücke benötigt. Dazu sind in Datalog spezielle vordefinierte Prädikate vorhanden. Für jeden Vergleichsoperator wird die Existenz eines solchen Prädikates angenommen.\\
Zunächst: Beschränkte Variablen in Regeln. Sei eine Regel r gegeben:
\begin{itemize}
\item Jede Variable, die als Argument in einem gewöhnlichen Prädikat im Rumpf von r vorkommt ist beschränkt.
\item Jede Variable, die in einem Teilziel $X=c$ oder $c=X$ von r vorkommt, ist beschränkt.
\item Eine Variable X ist beschränkt, wenn sie in einem Teilziel $X=Y$ oder $Y=X$ von r vorkommt mit Y ist schon als beschränkt bekannt.
\end{itemize}

\subsubsection*{Definition: sicher}
Eine Regel heißt sicher, wenn alle in ihr vorkommenden Variablen beschränkt sind.

\paragraph*{Beispiele:}
\begin{itemize}
\item $Kls(X,Y) :- bl(Z, s, j), ag(Z, Y), X=Z.$ \textbf{sicher}
\item $vsj(X,Y) :- bl(Y,s,j).$ \textbf{nicht sicher} (X ist nicht beschränkt)
\item $vs(X,Y) :- vs(X, Z), kp(Z, Y).$ \textbf{sicher, wenn vs terminiert}
\item $kla(Z,Y) :- bl(Z,V,j), ag(Z,Y), V \neq s.$ \textbf{sicher}
\end{itemize}

\paragraph*{Bemerkung:} Falls keine Build-in Prädikate erlaubt sind (/vorkommen):\\
Eine Regel ist sicher genau dann wenn jede Variable im Kopf der Regel auch im Rumpf der Regel vorkommt.

\subsubsection*{Definition: Datalog Programm}
Ein Datalog-Programm P (ohne IBen(Integritätsbedingungen)) ist eine endliche Menge von Horn-Klauseln mit Jedes $d \in P$ ist entweder
\begin{itemize}
\item ein Fakt $q(...).$ ohne Variable
\item eine sichere Regel $q(...) :- p_1(...),...,p_n(...).$ mit $q\in iPraedikat$
\end{itemize}

Ein $d \in P$ heißt auch \textbf{Datalog-Klausel}
Alle Fakten zu extensionalen Prädikaten sind als in DB-Relationen gespeichert zu denken.

\paragraph*{Beispiel}
Datenbankzustand:

\begin{equation}
\begin{split}
&ag(a1, m).\\
&kp(c2, a0).\\
&... \\
&rb(a1, r1, t1) \textit{(Kurs a1 im Raum r1 zu t1)}\\
&rb(a3,r2,t4)
\end{split}
\end{equation}

Angebot: \\
\begin{tabular}{|c|c|}
\hline
Kursnummer & Dozent\\ \hline
a1 & m \\
... & ... \\
\hline
\end{tabular}

Kursplan: \\
\begin{tabular}{|c|c|}
\hline
Kursnummer & Voraussetzung\\ \hline
c2 & a0 \\
... & ... \\
\hline
\end{tabular}

Raumbelegung: \\
\begin{tabular}{|c|c|c|}
\hline
Kursnummer & Raum & Zeit\\ \hline
a1 & r1 & t1 \\
... & ... \\ & ... \\
\hline
\end{tabular}

Belegung: \\
\begin{tabular}{|c|c|c|}
\hline
Kursnummer & Teilnehmer & Wiederholer\\ \hline
a1 & s & j \\
... & ... \\ & ... \\
\hline
\end{tabular}

\begin{equation}
\begin{split}
vs(X, Y) &:- Kp(X,Y). \\
vs(X, Y) &:- vs(X, Z), Kp(Z, Y). \\
stdpl(W,X,Y,Z) &:- bl(X, W, V), rb(X,Y,Z). \\
ueberschneidungen(X,Y) &:- Kp(Z, X), Kp(Z, Y), rb(X, V,T), rb(Y, W, T), X \neq Y.
\end{split}
\end{equation}


\subsubsection*{Deklarative Semantik}
Extensionale Prädikate eines Programms (ext. Rel, ext. DB): EDB\\
Intentionale Prädikate eines Programms (int. Rel, int. DB): IDB\\

\paragraph*{Bedeutung}
Bedeutung eines Datalog-Programms P: Menge derjenigen Grundatome zu den intentionalen Prädikaten von P, die logisch aus P gefolgert werden können. (Jedes Modell von P ist auch ein Modell von $f \in F$). Mit Zielklausel g(...), $g \in Praed$: Aus P logisch folgerbare Grundatome zu g, die von g(..) subsummiert (überdeckt) werden.
\begin{tabular}{c}
g(a,X) \\ \hline
g(a,b) \\
g(a,c)
\end{tabular}

In P werden Werte aus Wertebereichen verwendet, ebenso in Darstellung der extensionalen Prädikate als DB-Relation. Daher können wir $Kost_A$ und Dom idefntifizieren. Mithilfer der Herbrand Interpretation kann die Semantik festgelegt werden (ist möglich).


\paragraph*{Herbrand-Interpretation /-Modelle}

Gewöhnliche Interpretation: \\
\begin{equation}
\begin{split}
Konst_A &= \{ a,b \}, Dom = \{ \circ, \square \} \\
k(a) &= \circ\\
k(b) &= \square \\
ext(p(.,.)) &=  \{ (\circ, \square), (\square, \square) 
\end{split}
\end{equation}

eine mögliche Herbrand-Interpretation (passt dazu)
\begin{equation}
\begin{split}
Konst_A &= Dom = \{ a,b \} \\
k(a) &= a\\
k(b) &= b \\
ext(p(.,.)) &=  \{ (a, b), (b, b) 
\end{split}
\end{equation}

Entsprechende Herbrand-Interpretation. Betrachte alle Paare zu p(.,.), teste gemäß gegebener (gewöhnlicher) Interpretationin ext(p(.,.)).
\begin{equation}
\begin{split}
Konst_A &= \{ a,b \}, Dom = \{ \circ, \square \} \\
k(a) &= \square\\
k(b) &= \square \\
ext(p(.,.)) &=  \{ (\circ, \square), (\square, \square) 
\end{split}
\end{equation}
(a,a) wird zu $(\square, \square) \in ext(p(.,.))$

Herbrand-Interpretation
\begin{equation}
\begin{split}
Konst_A &= Dom = \{ a,b \} \\
k(a) &= a\\
k(b) &= b \\
ext(p(.,.)) &=  \{ (a,a), (a, b), (b,a), (b,b) \}
\end{split}
\end{equation}

Bei beiden Interpretationen sind die gleichen Formeln gültig bei Beschränkung auf quantorenfreie Formeln ohne Variablen (und ohne Funktionen).

\paragraph*{Beispiel}

Erste Interpretation: $p(a,b) \wedge p(b,a) \Rightarrow (\square, \square) \in ext(p(.,.)) \wedge ...$ bzw. $(a,b) \in ext(p(.,.)) \wedge (b,a) \in ext(p(.,.))$

Menge von Konstanten und Prädikatensymbole ist endlich, daher ist die Anzahl der möglichen Herbrand-Interpreationen endlich.

\subsubsection*{Satz von Gödel / Skolem (Herbrand, 1930)}
Eine Klauselmenge P hat ein Modell genau dann wenn P hat ein Herbrand-Modell.
Daraus folgt, dass ein Verfahren analog zu Wahrheitstabellen in der Aussagenlogik möglich ist.

\paragraph*{Beispiel}
$F = \{ p(a) \Rightarrow q(b), p(a) \wedge q(b) \}$, $q(b)?$

\begin{tabular}{|c|c|c|c|c|}
\hline
p & q & $p(a) \Rightarrow q(b) \text{erfüllt?}$ & $p(a) \wedge q(b) \text{erfüllt?}$ & $p(a) \Rightarrow q(b) \text{und} p(a) \wedge q(b) \text{erfüllt?}$\\ \hline
$\{\}$ & $\{\}$ & \checkmark & - & - \\
$\{\}$ & $\{ a \}$ & \checkmark & - & - \\
$\{\}$ & $\{ b \}$ & \checkmark & \checkmark & \checkmark \\
$\{\}$ & $\{ a, b \}$ & \checkmark & \checkmark & \checkmark \\
$\{ a \}$ & $\{\}$ & - & \checkmark & - \\
... & ... & ... & ... & .... \\
$\{ b \}$ & $\{ b \}$ & \checkmark & \checkmark & \checkmark
\end{tabular}

Jedes Modell von F ist auch ein Modell von q(b), d.h. q(b) kann aus F logisch gefolgert werden. Gilt bei Klauselmengen, aber \textbf{Vorsicht bei allgemeinen Formeln}.
\paragraph*{Beispiel:} $\{p(a), (\exists X) (\lnot p(X))\}$ Formelmenge, keine Klauselmenge

Modell (vgl. Übung): 
\begin{equation}
\begin{split}
Dom &= \{ 0, 1 \} \\
k(a) &= 0\\
ext(p(.)) &=  \{ (0) \} 
\end{split}
\end{equation}

\paragraph*{Aber:} Es gibt kein durch \texttt{ext} bestimmtes Herbrand-Modell:

\begin{enumerate}
\item $ext(p(.)) = \{(a)\}, Konst = Dom = \{ a \}$
\item $ext(p(.) = \{\})$
\end{enumerate}

Herbrand-Modell muss genügend viele Elemente enthalten, damit der Satz von Gödel / Skolem gelten kann. \textbf{Skolemisierung} bedeutet, dass man alle Existenzquantoren durch Funktionen ersetzt:

\begin{equation}
(\forall x_1,....,x_n)(\exists y) (F) \leadsto (\forall x_1,...,x_n)(F[f(x_1,...x_n) / y])
\end{equation}


\subsubsection*{Bemerkung: Skolemisierung}

Jede Formel der PL1 Logik kann man in einer \underline{erfüllbarkeitsäquivalente} Formel in Skolemform umformen:
\begin{enumerate}
\item Pränexnormalform
\item Umformungen à la $(\forall x_1,....,x_n)(\exists y) (F) \leadsto (\forall x_1,...,x_n)(F[f(x_1,...x_n) / y])$ mit jeweils einem neuen Funktionssymbol
\end{enumerate}
Dies ist eine Art ``Materialisierung'' der durch den Existenzquantor gebundenen Variablen.

\paragraph*{Beispiel (von oben)}

$\{p(a), \lnot p(y)\footnote{neue Variable} \}$ erfüllbar $\Longleftrightarrow \{p(a), (\exists X)(\lnot p(X))\}$ erfüllbar \\

\textbf{Vorsicht:} Semantische Äquivalenz von Formeln und ihren Skolem-Normalformen im Allgemeinen bicht gegeben.

Skolem-NF: $(\exists X)(p(X)) : p(a)$
Interpretation I mit 
\begin{equation}
\begin{split}
Dom &= \{ 1, 2 \} \\
k(a) &= 1\\
ext(p(.)) &=  \{ (2) \} 
\end{split}
\end{equation}

$\leadsto \vDash_I (\exists X)(p(X))$
Belegung von X mit allen Elementen aus Dom, d.h. auch mit 2. Aber $\not \vDash_I p(a)$, da $(1) \not \in ext(p(.))$.

\subsubsection*{Im Kontext von Datalog}
\begin{itemize}
\item Herbrand-Universum: Konst
\item Herbrand-Basis (HB): Menge aller Grundatome \\ ($\leadsto EDB\footnote{gegeben durch Datenbankzustand}, IDB\footnote{muss ausgerechnet / gefolgert werden} \text{(Fakten in der Datenbank und solche die Ableitbar sind)} \subset HB$ )
\item Herbrand-Interpretation: (Konst, id, ext), d.h. jedes Konstantensymbol wird als es selbst interpretiert. (Verglichen mit relationaler Interpretation, dort $k$ ist Bijektion)
\item Jede Herbrand-Interpretation ist eindeutig bestimmt durch $ext$ (Extension, Ausprägung), da Konst und id unveränderlich sind
\item Jedes Prädikat ist eindeutig bestimmt durch die Angabe der Grundatome, für  die es ``wahr'' liefert.
\item extentional: genaue Antwort in Tupel
\item intentional: Beispielsweise Formeln als Antworten
\end{itemize}

\subsubsection*{Definition: Herbrand-Interpretation}
Einfache Definition: Eine Herbrand-Interpretation ist eine Teilmenge der Herbrand-Basis.


\paragraph*{Beispiel Kursverwaltung}

\begin{equation}
\begin{split}
\text{Aus DB-Rel. KURSPLAN} &\leadsto kp \in ePr\ddot{a}d \\
vs(X,Y) :- kp(X,Y) &\leadsto vs \in iPr\ddot{a}d
\end{split}
\end{equation}

\paragraph*{Gesucht:} Durch Programm P bestimmte Grundatome (Fakten).

\subsubsection*{Definition: Grundatom}
Ein Grundatom f ist eine logische folgerung einer Menge D von Datalog Klauseln (z.B. $D \vDash f$) $\diamondsuit_{Def}$. Jedes Herbrand Modell von D ist auch ein Modell von f.\\
Da f ein Grundatom ist gilt $D \vDash f \Longrightarrow f $ ist in jedem Herbrand.Modell von D enthalten. Das heißt $f \in \bigcap \{ I | I Herbrand-Modell von D \}$.\\
Sei $f \in \bigcap \{ I | I Herbrand-Modell von D \}$, dann ist f ein Grundatom und jedes Modell von D auch in Modell von f. 

\subsubsection*{Definition: Mege aller Konsequenzen}

$cons(D) =_{def} \{ f \in HB_D | D \vDash f \}$


\subsection*{Satz 1.1}

$cons(D) = \bigcap \{ I | I Herbrand-Modell von D \}$ \\
Aufgrund der Eigenschaften unserer Regel ist der Schnitt / cons(D) ein Modell, dies gilt es zu beweisen: \\

Da $cons(D) \subseteq HB_D$, ist $cons(D)$ eine Herbrand-Interpretation.

\subsection*{Satz 1.2} 

$cons(D)$ ist ein Herbrand-Modell von D.

\subsubsection*{Beweis}
\paragraph*{z.Z.} 
Jedes $d \in D$ ist gültig in dieser Interpretation, also $cons(D)$ ($\vDash_{cons(D)} d$).

\paragraph*{Beweis}

Falls $d$ ein Grundatom ist gehört $d$ zu jedem Herbrandmodell von D, daraus folgt $d \in cons(D)$. \\
Sei d eine Regel $q(...) :- p_1(...),...,p_m(...)$. Sei $\varrho$ eine Belegung für die Variable von d. \textit{Annahme:} $(\forall 1 \le i \le m)(|| p_i(.)||^{\varrho} \in cons(D))$, sonst d gültig unter $cons(D)$.
Für jedes Herbrand-Modell I von D gilt, dann $||p_i(.)||^{\varrho} \in I, i = 1, ..., m$ und damit $||q_(.)||^{\varrho} \in cons(D),$ da $d$ eine Horn-Klausel ist.
(Der Schluss ist z.B. für $d = q_1(...), q_2(...) :- p_1(...), ..., p_n(...)$ nicht möglich.) \\

Damit $D \vDash_{cons(D)} ||q(...)||^{\varrho} \in cons(D)$. Wir erhalten, dass $cons(D)$ ein Herbrand-Modell von D ist. Da $d$ beliebig gewählt wird folgt der Satz.

$cons(D)$ ist offensichtlich eindutig bestimmt und das kleinste Herbrand-Modell von D. Damit: Seantik eines Datalog-Programms P ist gegeben durch das kleinste Herbrand-Modell von P oder (äquivalent) durch $cons(P) = \{ f\in HB_{P} | O \vDash f \}$.

\paragraph*{Beispiel}
$r \in ePr\ddot{a}d, p,q \in iPr\ddot{a}d, Konst = \{ 1, 2, 3 \}$.

\textbf{Fakten}: \\
\begin{equation}
\begin{split}
r(1).& \\
p(X) &:- q(X). \\
q(X) &:- r(X).
\end{split}
\end{equation}

\textbf{Herbrand-Modelle}: \\
\begin{equation}
\begin{split}
\{ r(1), p(1)&, q(1) \} \\
\{ r(1), p(1)&, q(1), q(2), p(2) \} \\
\{ r(1), p(1)&, q(1), q(2), p(2), p(3) \} \\
\end{split}
\end{equation}

\subsection*{Fixpunkt-Semantik}
Deklarative Semantik liefert keinen brauchbaren Algorithmus. Die Fixpunkt-Sematik führt direkt zu einem algorithmus für die Schrittweise Berechnung des kleinsten Herbrand-Modells.


%insert img1.png here
\begin{equation}
\begin{split}
&c \in ac \\
&\tau(c) \subseteq \tau(ac) \\
&\tau(c) = c ! \\ % TODO: add cancel
&\tau(c) = ac
\end{split}
\end{equation}

\begin{equation}
\begin{split}
&c \in bc \\
&\tau(c) \subseteq \tau(bc) \\
&\tau(c) = c ! \\ % TODO: add cancel
&\tau(c) = bc
\end{split}
\end{equation}

Dies ist ein Wiederspruch


\paragraph{P Datalog-Programm}
$\tau_P$ liefert Fakten von P vereinigt mit Fakten, die in einem Schritt aus der Argumentenmenge von $\tau_P$ mit den Regeln von P abgeleitet werden können.

\subsection*{Ableitung in einem Schritt}

\subsubsection*{Definition: Substitution}
Eine Substitution ist eine endliche Menge der Form
\begin{equation}
\{ X_1 / t_1, \cdots, X_n / t_n \}, X_1,...,X_n unterschiedliche Variablen, t_1,....,t_n Terme, X_i \neq t_i
\end{equation}

Falls alle $t_i$ Konstanten ist dies eine \textbf{Grundsubstitution}.

Sei $\theta$ eine Substitution, t ein Term (Variable oder Konstante), so gilt \\

\begin{equation}
t\theta =_{def} \begin{cases} t_i, & \mbox{falls } t/t_i \in \theta \\ t, & \mbox{sonst} \end{cases}
\end{equation}
Sei L ein Literal, $L\theta$ bezeichnet dasjenige Literal, das aus L entsteht, indem alle Variablen $X_i$ mit $L_i$ für die $X_i / t_i \in \theta$ gilt, simultan durch $t_i$ ersetzt werden. Analog $d\theta$ für eine Datalog-Klausel d.

\paragraph{Beispiel}

\begin{equation}
L = p(X, Y, a), \theta = \{X / a, Y / X \} \\
L\theta = p(a, X, a)
\end{equation}

Sei D eine Menge von Datalog Klauseln. In einem Schritt aus D ableitbare Menge von Fakten:

\begin{equation}
fakt_1(D) = \{ f \in HB_D | (\exists Regel L_0 :- L_1,...,L_n)(\exists Sicht \theta)(\{ L_1 \theta, \cdots, L_n\theta \} \subseteq D \wedge f = L_0\theta) \}
\end{equation}

(Annahme: built-in Prädikate geeignet berücksichtigt $\Rightarrow$ Algorithmus FAKT\_1)

Definiere $\tau_P | 2^{HB} \Rightarrow 2^{HB}$. $\tau_P(I) =_{def} P_F \cup fakt_1(P_R \cup I)$.$ P_f = $ Menge der Fakten von P. $P_P$ Menge von Regeln von P.

\subsection*{Satz 1.3} Für jedes Datalog-Programm P ist $\tau_P$ eine monotone Transformation auf $(2^{HB}, \subseteq)$.

\subsubsection*{Beweis}
Seien $I_1, I_2 \in HB_P$ mit $I_1 \subseteq I_2$. z.Z.: $\tau_P(I_1) \subseteq \tau_P(I_2)$. \\
Sei $f \in \tau_P(I_1)$. Falls $f \in P_F$, dann gilt auf $f \in \tau_p(I_2)$. $f \in fakt_1(P_R \cup _1)$ da $I_1 \subseteq I_2$, gilt $P_R \cup I_1 \subseteq P_R \cup I_2$ und somit $f \in fakt_1(P_R \cup I_2)$ wg. Monotonie von Fakt\_1 auf ($2^{HB}, \subseteq$).


\paragraph{Beispiel}

\begin{equation}
\begin{split}
P = &p(1) \\
= &p(2). \\
= &r(1). \\
&\\
q(X) :- &s(X), r(X). \\
s(X) :- &p(X).
\end{split}
\end{equation}

$\leadsto^{T_p(\emptyset)}$

\begin{equation}
\begin{split}
&p_1\\
&p_2 \\
&r_1
\end{split}
\end{equation}

$\leadsto^{T_p(\cdots)}$

\begin{equation}
\begin{split}
&p_1\\
&p_2 \\
&r_1\\
&s(1) (\text{fakt\_1}) \\
&s(2) (\text{fakt\_1})
\end{split}
\end{equation}

$\leadsto^{T_p(\cdots)}$

\begin{equation}
\begin{split}
&p_1\\
&p_2 \\
&r_1\\
&s(1) \\
&s(2) \\
&q(1) (\text{fakt\_1})
\end{split}
\end{equation}

\subsection*{Satz 1.4} Für jedes Datalog-Programm P gilt $cons(P) = lf_p(\tau_p)$ (lf = least fixpunkt).

\subsubsection*{Beweis}:
\paragraph{1)} $cons(P) ist ein Fixpunkt von \tau_P$. \\
$\tau_p(cons(P)) = P_F \cup fakt\_1(P_R \cup cons(P))$. cons(P) ist kleinstes Herbrand-Modell von P, d.h. alle Regeln sind gültig unter $cons(P) \Rightarrow fakt\_1(P_R \cup cons(P) = cons(P) \backslash P_F)$. (und kein Fakt kann aus cons(P) entfernt werden, ohne dass die Modelliereigenschaft verloren geht.) $\Rightarrow \tau_P(cons(P)) = P_F \cup cons(P) \backslash P_F = cons(P)$

\paragraph{2)} $cons(P)$ ist minimaler Fixpunkt von $\tau_P$.
Annahme: $(\exists I \cancel{\subseteq} cons(P))(\tau_P(I) = I)$.
cons(P) ist minimales Herbrand-Modell $\Rightarrow$ I ist kein Herbrand-Modell. Da $P_F \subseteq I$ wg. Annahme $\tau_P(I) = I$ folgt midestens eine Regel ist nicht erfüllt, d.h. $(\exists h_0, \cdot,h_n \in P_R)(\exists Substitution \theta)(\{ h_1\theta, \cdots, L_n(\theta)\} \subseteq I \wedge L_1 \theta \cancel{\in} I)$. Aber $L_0 \in fakt_1(P_R \cup I)$ nach Definitio von fakt\_1.
Da auch $L_0 \theta \in P_F$ wegen $P_F \subseteq I, folgt \tau_P(I) \neq I$. Noch Fixpunkttheorem (Kuaster / Tarski) ist minimaler Fixpuknt von $\tau_P$ kleinster Fixpunkt von $\tau_P$. 

\subsection*{Vorgehensweise bei Fixpunktberechnung} bottom-up

\subsubsection*{Bezeichnung: Forward-chaining}Da ``$\Leftarrow$'' die natürliche Richtung für die Anwendung von Regeln ist.
Da Regeln sicher sind ist $L_i \theta$ stets ein Fakt $\theta_i$.
Betrachte Eignung der Methode zur Anfrageauswertung, z.B. ?- vs(c4, y). ``Alle Kurse, die Voraussetzung von Kurs c4 sind''

\paragraph{Ineffizient} Berechnung des $lf_p$ und anschließende Siche zu vs(c4, y) passende Fakten $\Rightarrow$ Vermeide möglichst Berechnung nicht relevanter Fakten (später).

Starte mit Zielklausel, Suche ein Regel die zu einem Atomder Klausel passt. Ersetze das Atom durch angepassten Rumpf der gewählten Regel, neue Anfrage, iterieren.

\subsubsection*{Bezeichnung: Backward-chaining}
\paragraph{Beispiel}
\begin{equation}
\begin{split}
P = &p(1) \\
= &p(2). \\
= &r(1). \\
&\\
q(X) :- &s(X), r(X). \\
s(X) :- &p(X).
\end{split}
\end{equation}

% img 3 hier


\subsection*{Prozedurale Semantik}
Beweistheoretische Sicht, übertragen auf Datalog Programm P aufgefasst als Theorie 1. Stufe \\
Semantik: $\{f \in HB_P | P \vdash f \}$ \\
Wie können Ableitungen (Beweise) von Fakten systematisch durchgeführt werden?

Regeln: 
\begin{equation}
\begin{split}
(1) vs(X, Y) &:- Kp(X, Y). \\
(2) vs(X, Y) &:- vs(X, Z), kp(Z, Y). \\
P &= \{ Kp (c4, a3), \cdots, kp(c2, a0), (1), (2) \} 
\end{split}
\end{equation}

Zeige $P \vdash vs(c4, a0)$. 
\begin{equation}
\begin{split}
(1) = (\forall X)(\forall Y)&(kp(X, Y) \Rightarrow vs(X, Y)) \\
(2) = (\forall x)(\forall Y)(\forall Z)&(vs(X, Z) \wedge kp(Z , Y) \Rightarrow vs(X, Y))
\end{split}
\end{equation}

\paragraph{Beweis}
% img 4 hier 
% img 5 hier

\paragraph{Idee} 

\begin{itemize}
\item Erzeige alle Beweismister (Suchbäume)
\item Suche Belegungen für die Variable, so dass alle Blätter zu Fakten aus P werden
\item jede solche Belegung erzeugt einen Beweisbaum.
\end{itemize}

\paragraph{Zunächst} Suche nach ``passenden'' Köpfen von Regeln erfordert Definition.

\subsubsection*{Definition: Unifizierbar}

Seien $L_1$ und $L_2$ heißen \textbf{unifizierbar}, wenn $(\exists \text{ Substitution } \Theta)(L_1\Theta = L_2\Theta)$. $\Theta$ heißt dann \textbf{Unifikator}.

\paragraph{Beispiel}

$L_1 = vs(X, Z)$ \\
$L_2 = cs(X, Y)$ \\
$\Theta = \{ Z / Y \}$ und $\Theta = \{ Y / Z \}$ sind Unifikatoren von dem Paar $L_1, L_2$, aber auch $\Theta = \{ X/a, Y/a, Z/a \}$. Die ersten beiden sind spezifischer als das letzte

$L_1 = q(X, Y ,c) L_2 = q(W, b, Z)$
Unifikation z.B.:
\begin{equation}
\begin{split}
\Theta_1 &= \{ X/W Y/b Z/c \} \\
\Theta_2 &= \{ X/T, W/T, y/b, Z/c \} \\
\Theta_3 &= \{ X/a, W/a, Y/b, Z/c\}
\end{split}
\end{equation}

Nicht unifizierbar:
$L_1 = q(X, c, U) L_2 = q(W, G, Z)$ oder $L_1 = q(X, a, X) L_2 = q(b, Y, Y)$

\subsubsection*{Definition: Komposition}

Sei $\Theta = \{ X_1 / t_1, \cdots, X_n / t_n \}, \varsigma = \{ Y_1 / n_1, \cdots, Y_m / t_m \}$ Substitutionen. \\
Die Komposition $\Theta\varsigma$ von $\Theta$ und $\varsigma$ erhält man aus 
\begin{equation}
X_1 / t_1\varsigma, \cdots, X_m / t_m\varsigma, Y_1 / n_q, \cdots, Y_m / n_m
\end{equation}
Durch Streichen von Elementedn der Form Z/Z sowie $Y_i / n_i$ mit $Y_i = X_j$ für ein j$j \in \{1, ..., n\}$

\paragraph{Beispiel} $\theta= \{ X/a,  Y/W \} \varsigma=\{X/ bm Y/ V, W/Z \}$ \\
$\Theta\varsigma = \{ X/a, YZ, W/Z \}$

\subsubsection*{Definition: allgemeinere Substitution}
Seien $\Theta, \varsigma$ Substitutionen, $\Theta$ heißt allgemeiner als $\varsigma \diamondsuit (\exists \text{Substitution} \delta)(\Theta \delta = \varsigma)$.
Seine $L_1, L_2$ Literale. Ein allgemeinster Unifikator (mgu) von $L_1 in L_2$ ist ein Unifikator von $L_1 und L_2$, der allgemeiner als alle anderen Unifikatoren ist.

\paragraph{Beispiel} $\Theta_2$  ist allgemeiner als $\Theta_3$; betrachte $\delta = \{T / a\}$, es gilt $\Theta_2\delta = \Theta_3$. $\Theta_1$ ist allgemeiner als $\Theta_2, \delta = \{ W / T \}$. mgu ist i.A. nicht eindeutig bestimmt. $L_1 = p(X,X) L_2 = p(V,W)$
mgu: \\
\begin{equation}
\begin{split}
&\{ X/W, V/W \} \\
&\{ X / V, W/V\} \\
&\{ V/X, W/X \}
\end{split}
\end{equation}

\paragraph{Beispiel}
$L_1 = q(X,Y,c)\footnote{t1, t2, t3} L_2 = q(W, b, Z) \footnote{k1 k2 k3}$ \\
% bild 5 hier einfügen


\subsection*{Suchbaum} (Beweismuster zu einem Programm P)
\begin{itemize}
\item Wurzel ist mit einem ``Ziel'' $g = p(t_1, \cdots, t_n), p \in iPr\ddots{a}d$, bekannt.
\item Knoten entlang eines Pfases von der Wurzel aus sind abwechselnd mit Atome und Regeln benannt. (Ziel-, Regelknoten)
\item Alle Blattknoten sind mit Atomen benannt
\item Sei k ein mit einem Atom $a(s_1,...,s_l)$ benannten Knoten (keine Blattknoten), dann ist der unmittelbare Nachfolger von k mit einer Regel von P benannt, deren Kopf mit $q(sq, ..., s_l)$ unifizierbar ist
\item sei k' ein mit einer Regel $r = L_0 :- L_1,...,L_m$ benannte Knoten der unmittelbare Vorgänger von k' sei mit $q(s_1,...,s_l)$ benannt. Dann hat k' m unmittelbare Nachfolger, die mit $I_1 mgu(q(s_1, ..., s_l), I_{0}^{~}), ..., I_m mgu(q(s_1,...,s_l), I_0)$ benannt sind. Dabei sind $I_0, .., I_m$ Atome, die aus $L_1,...,L_m$ durch eventuelle Umbenennung von Variablen hevorgehen (die Variablen seien stets so umbenannt, dass sie im Baum eindeutig sind).
\end{itemize}

\paragraph{Beispiel}
% bild 6 hier


\end{document}