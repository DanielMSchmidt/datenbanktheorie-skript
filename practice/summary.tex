\documentclass[12pt, a4paper]{article}
\usepackage{url,graphicx,tabularx,array,geometry}
\usepackage[utf8]{inputenc}
\usepackage{paralist}
\usepackage{latexsym}
\usepackage{fancyhdr}
\usepackage{ textcomp }
\usepackage{ mathrsfs }
\usepackage{amssymb}

\pagestyle{fancy}

\usepackage{amsmath}
\usepackage{amsfonts}
\usepackage{amssymb}

\DeclareUnicodeCharacter{00A0}{ }

\setlength{\parskip}{1ex} %--skip lines between paragraphs
\setlength{\parindent}{0pt} %--don't indent paragraphs

%-- Commands for header
\newcommand{\bs}{\ensuremath{\backslash}}
\renewcommand{\title}[1]{\textbf{#1}\\}
\renewcommand{\line}{\begin{tabularx}{\textwidth}{X>{\raggedleft}X}\hline\\\end{tabularx}\\[-0.5cm]}
\newcommand{\leftright}[2]{\begin{tabularx}{\textwidth}{X>{\raggedleft}X}#1%
& #2\\\end{tabularx}\\[-0.5cm]}
%\linespread{2} %-- Uncomment for Double Space
\begin{document}
\renewcommand{\headrulewidth}{0pt}
\fancyhf{}
\fancyhead[L]{
\leftright{\textbf{Zusammenfassung der Übung}}{Daniel Schmidt}
\line
\leftright{\textbf{Datenbanktheorie SS 16}}{}}
\fancyfoot[C]{\thepage}

\section*{Übung 3}

\subsection*{Aufgabe 1}

$T_1: T_0 + $ Abschlussaxiom für Wertebereich + Gleichheitsaxiome\\

\subsubsection*{a)}
$/lnot = (a1, c4)$\\
Gib Interpretation I an, die Modell von $T_1$ ist, aber nicht von $\lnot = (a1, c4) \leadsto \lnot = (a1, c4)$ nicht ableitbar (Satz von Gödel).\\

\paragraph*{I.}
$Dom = Konst_A$ als Beispiel, ext wie angegeben ext(=) wie üblich. $K: Konst_A \Rightarrow Dom$ mit $k(c) = \begin{cases} a1, & \mbox{wenn } c = c4\\ c, & \mbox{sonst} \end{cases}$. \\

Nach Konstruktion gilt:
\begin{itemize}
\item $T_0$ erfüllt
\item Gleichheitsaxiom erfüllt, insbesondere $=(a1, c4)$, da $k(c4) = a1$
\item Abschlussaxiom erfüllt
\end{itemize}

Aber 
\begin{equation}
\begin{split}
\nvDash_I \lnot &= (a1, c4)\\
\diamondsuit nicht \vDash_i \lnot &= (a1, c4)\\
\diamondsuit nicht nicht \vDash_i &= (a1, c4)\\
\diamondsuit \vDash_i &= (a1, c4)
\end{split}
\end{equation}

\subsubsection*{b)}
$\lnot do(a0)$\\
\paragraph*{I.} Dom wie oben, k = id, $ext(do) = \{ m, q, d, a0 \}$\\
I erfüllt $T_2$, Gleichheits, Abschluss und Eindeutigkeitsaxiome

Aber 

\begin{equation}
\begin{split}
\nvDash_I &\lnot do(a0)\\
\diamondsuit nicht \vDash_I &\lnot do(a0)\\
\diamondsuit nicht nicht \vDash_I &do(a0)\\
\diamondsuit \vDash_I &do(a0)\\
||a0||^\rho_I &= (a0) \in ext(do)
\end{split}
\end{equation}

\subsection*{Aufgabe 2}

\paragraph*{$f_1$}: Sei $\rho$ eine Belegung mit $\rho(X) = einbruch, \rho(Z) = pathologie$. Dann gilt $(\rho(X), offen, \rho(Z)) \in ext(akte)$\\
Also gilt: 

\begin{equation}
\begin{split}
\vDash_{I, \rho}& akte(X, offen, Z)\\
nicht nicht \vDash_{I, \rho}& akte(X, offen, Z)\\
nicht \vDash_{I, \rho}& \lnot akte(X, offen, Z)\\
nicht \vDash_{I, \rho}& (\forall Z)(\lnot akte(X, offen, Z))\\
\vDash_{I, \rho}& \lnot (\forall Z)(\lnot akte(X, offen, Z))\\
\vDash_{I, \rho}&  (\exists Z)(akte(X, offen, Z))\\
\vDash_{I, \rho}& (\exists X)(\exists Z)(akte(x,offen,Z))
\end{split}
\end{equation}
analog mit X
\\
\paragraph*{$f_2$}: Sei $f':_2 = le(X_1, Y) \wedge le(X_2, Y)$.\\
Sei $f''_2 = (X_1, X_2)$\\
Sei $\rho$ eine Belegung.

\textbf{1.Fall}\\
\begin{equation}
\begin{split}
\rho(X_1) &= \rho(X_2)\\
\vDash_{I, \rho}& f'_2\\
\vDash_{I, \rho}& \lnot f'_2 \vee f''_2\\
\vDash_{I, \rho}& f'_2 \Rightarrow f''_2\\
(\forall X_1)(\forall X_2)(\forall Y)(le(X_1, Y) \wedge le(X_2, Y) \Rightarrow (X_1, X_2)
\end{split}
\end{equation}

\textbf{2. Fall} $\rho(X_1) = \rho(X_2)$\\
Es gilt: $ext(le) = \{ (skinner, mulder), (skinner, scully) \}$\\

Mit $\rho(X_1) = \rho(X_2)$ folgt:

\begin{equation}
\begin{split}
&nicht (X_1, Y) \in ext(le) \vee nicht (X_2, Y) \in ext(le)\\
&nicht ((X_1, Y) \in ext(le) \wedge (X_2, Y) \in ext(le)\\
&nicht \vDash_{I, \rho} f'_2\\
&\vDash_{I, \rho} \lnot f'_2\\
&\vDash_{I, \rho} \lnot f'_2 \vee f''_2\\
&\vDash_{I, \rho} \lnot (\forall X_1)(\forall X_2)(\forall Y)(le(X_1, Y) \wedge le(X_2, Y) \Rightarrow = (X_1, X_2))
\end{split}
\end{equation}


\subsection*{Aufgabe 3}

$f_1$: 
\begin{equation}
\begin{split}
&\dash akte(krycek, offen, psychologie)\footnote{mit Axiom}\\
&\dash (\exists Z) (akte(krycek, offen, Z))\\
&\dash (\exists X) (\exists Z) (akte(X, offen, Z))
\end{split}
\end{equation}

$f_2$: Es sei $K_{le} = \{ (skinner, mulder), (skinner scully) \}$\\
1) $(\forall X_1)(\forall Y_1)(le(X_1, Y_1) \Rightarrow (=(X_1, skinner) \wedge =(Y_1, mulder)) \vee (= (X_1, skinner) \wedge =(Y_1, scully))$\\
2) $(\forall X_2)(\forall Y_2)(le(X_2, Y_2) \Rightarrow (=(X_2, skinner) \cdots$\\
3) $(\forall X_1)(\forall Y_1)(\forall X_2)(\forall Y_2)(le(X_1, Y_1) \wedge (le(X_2, Y_2) \Rightarrow (=(X_1, skinner) \wedge =(Y_1, mulder)) \vee (= (X_1, skinner) \wedge =(Y_1, scully)) \wedge (=(X_2, skinner) \wedge =(Y_2, mulder)) \vee (= (X_2, skinner) \wedge =(Y_2, scully))$\\
4) $(\forall X_1)(\forall Y_1)(\forall X_2)(\forall Y_2)(le(X_1, Y_1) \wedge (le(X_2, Y_2) \Rightarrow =(X_1, skinner) \vee (= (X_1, skinner) \wedge =(X_2, skinner) \vee = (X_2, skinner)$\\
5) $\cdots le(X_1, Y_1) \wedge le(X_2, Y_2) \wedge (Y_1 Y_2) \Rightarrow \cdots$\\
6) $(\forall X_1)(\forall X_2)(\forall Y)(le(X_1, Y) \wedge le (X_2, Y) \Rightarrow (=(X_1, skinner) \vee = (X_1, skinner) \wedge =(X_2, skinner) \vee =(X2, skinner))$\\
7) $\cdots \Rightarrow (=(X_1, skinner) \wedge =(X2, skinner))$\\
8) $\cdots \Rightarrow (=(X_1, skinner) \wedge =(skinner, X2))$\\
9) $(\forall X_1)(\forall X_2)(\forall Y)(le(X_1, Y) \wedge le(X_2, Y) \Rightarrow = (X_1, X_2))$

\subsection*{Aufgabe 4}
Überprüfen bei Existenzqauntor und implikation oder bei allquantor und keine implikation ensteht oft unfug\\

$(\forall s, w_1, f_1, w_2, f_2) ((STUDENT(s, w_1, f_1) \wedge STUDENT(s, w_2, f_2)) \Rightarrow STUDENT(s,w_1, f_2))$

\end{document}
