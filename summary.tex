\documentclass[12pt, a4paper]{article}
\usepackage{url,graphicx,tabularx,array,geometry}
\usepackage[utf8]{inputenc}
\usepackage{paralist}
\usepackage{latexsym}
\usepackage{fancyhdr}
\usepackage{ textcomp }
\usepackage{ mathrsfs }


\pagestyle{fancy}

\usepackage{amsmath}
\usepackage{amsfonts}
\usepackage{amssymb}



\setlength{\parskip}{1ex} %--skip lines between paragraphs
\setlength{\parindent}{0pt} %--don't indent paragraphs

%-- Commands for header
\newcommand{\bs}{\ensuremath{\backslash}}
\renewcommand{\title}[1]{\textbf{#1}\\}
\renewcommand{\line}{\begin{tabularx}{\textwidth}{X>{\raggedleft}X}\hline\\\end{tabularx}\\[-0.5cm]}
\newcommand{\leftright}[2]{\begin{tabularx}{\textwidth}{X>{\raggedleft}X}#1%
& #2\\\end{tabularx}\\[-0.5cm]}
%\linespread{2} %-- Uncomment for Double Space
\begin{document}
\renewcommand{\headrulewidth}{0pt}
\fancyhf{}
\fancyhead[L]{
\leftright{\textbf{Zusammenfassung}}{Daniel Schmidt}
\line
\leftright{\textbf{Datenbanktheorie SS 16}}{}}
\fancyfoot[C]{\thepage}

\section*{Deduktive Datenbanksysteme}

\textbf{Problem:} Transitiver Abschluss ist in PL1 nicht formulierbar (mit zustandsabhängiger Formulierung möglich)\\
\textbf{Diskussion:} \\
Typ 5: $q_1(...), ..., q_n(...) :- .$\\
Typ 6: $q_1(...),...,q_n(...) :- p_1(...),...,p_m(...).$\\
$\Rightarrow$ Übungsaufgabe

Nur Typ 1 und Typ 4: $q(...) :- p_1(...),...,p_n(...), n \ge 0$ (ist die Hornklauselform und wird bei definiten Datenbanken genutzt) 

\subsection*{Definite Datenbanken}
\begin{equation}
\begin{split}
q(...) &:- . \text{(Fakt)}\\
q(...) &:- p_1(...),...,p_n(...). \text{(deduktive Regel, } p_{1-n} \text{ Teilziele)}
\end{split}
\end{equation}

\begin{itemize}
\item Mit IBen (Integritätsbedingungen) (+ Typ2, Typ3) ($:- p_1(...), ...,p_n(...)$)
\item Typ5 + Typ6 $\Rightarrow$ \textbf{Disjunktive Datenbank}
\item Definite Datenbank + negative Atome im Rumpf von Hornklauseln erlaubt $\Rightarrow$ Volles Datalog
\end{itemize}

\subsubsection*{Formulierung von Anfragen}
Klauseln vom Typ: $:- p_1(...),....,p_n(...)$, geschrieben $? - p_1(...),....,p_n(...)$

Beispiele:
\begin{itemize}
\item $? - ag(X,m).$
\begin{itemize}
\item Bedeutung: Welche Kurse bietet 'm' an?
\item DRC: (x) / ANGEBOT(X, m) 
\end{itemize}
\item $? - ag(a3,m).$
\begin{itemize}
\item Bedeutung: Bietet 'm' den Kurs 'a3' an?
\item DRC: () / ANGEBOT(a3, m) 
\end{itemize}
\item $? - ag(X,m), bl(X,s,j).$
\begin{itemize}
\item Bedeutung: Gib alle von 'm' angebotene Kurse, die 's' als Wiederholer belegt hat
\item DRC: (x) / ANGEBOT(x, 'm') $\wedge$ BELEGUNG(x, 's', 'y') 
\end{itemize}
\item Wie ist (x) / ANGEBOT(x, 'm') $\wedge (\exists $y) BELEGUNG(x, 's', y) formulierbar?
\begin{itemize}
\item Bedeutung: Gib die Dozenten der von s als Wiederholer belegte Kurse
\item Formulierung: $? - Ksm(X). \\ Ksm(X) :- ag(X, m), bl(X,s,y)$
\item Bequemer: $?- ag(X, Y*), bl(X,s,y).$, * Kennzeichnet die Ausgabevariable
\end{itemize}
\end{itemize}

In Anfragesprachen werden Vergleichsausdrücke benötigt. Dazu sind in Datalog spezielle vordefinierte Prädikate vorhanden. Für jeden Vergleichsoperator wird die Existenz eines solchen Prädikates angenommen.\\
Zunächst: Beschränkte Variablen in Regeln. Sei eine Regel r gegeben:
\begin{itemize}
\item Jede Variable, die als Argument in einem gewöhnlichen Prädikat im Rumpf von r vorkommt ist beschränkt.
\item Jede Variable, die in einem Teilziel $X=c$ oder $c=X$ von r vorkommt, ist beschränkt.
\item Eine Variable X ist beschränkt, wenn sie in einem Teilziel $X=Y$ oder $Y=X$ von r vorkommt mit Y ist schon als beschränkt bekannt.
\end{itemize}

\subsubsection*{Definition: sicher}
Eine Regel heißt sicher, wenn alle in ihr vorkommenden Variablen beschränkt sind.

\paragraph*{Beispiele:}
\begin{itemize}
\item $Kls(X,Y) :- bl(Z, s, j), ag(Z, Y), X=Z.$ \textbf{sicher}
\item $vsj(X,Y) :- bl(Y,s,j).$ \textbf{nicht sicher} (X ist nicht beschränkt)
\item $vs(X,Y) :- vs(X, Z), kp(Z, Y).$ \textbf{sicher, wenn vs terminiert}
\item $kla(Z,Y) :- bl(Z,V,j), ag(Z,Y), V \neq s.$ \textbf{sicher}
\end{itemize}

\paragraph*{Bemerkung:} Falls keine Build-in Prädikate erlaubt sind (/vorkommen):\\
Eine Regel ist sicher genau dann wenn jede Variable im Kopf der Regel auch im Rumpf der Regel vorkommt.

\subsubsection*{Definition: Datalog Programm}
Ein Datalog-Programm P (ohne IBen(Integritätsbedingungen)) ist eine endliche Menge von Horn-Klauseln mit Jedes $d \in P$ ist entweder
\begin{itemize}
\item ein Fakt $q(...).$ ohne Variable
\item eine sichere Regel $q(...) :- p_1(...),...,p_n(...).$ mit $q\in iPraedikat$
\end{itemize}

Ein $d \in P$ heißt auch \textbf{Datalog-Klausel}
Alle Fakten zu extensionalen Prädikaten sind als in DB-Relationen gespeichert zu denken.

\paragraph*{Beispiel}
Datenbankzustand:

\begin{equation}
\begin{split}
&ag(a1, m).\\
&kp(c2, a0).\\
&... \\
&rb(a1, r1, t1) \textit{(Kurs a1 im Raum r1 zu t1)}\\
&rb(a3,r2,t4)
\end{split}
\end{equation}

Angebot: \\
\begin{tabular}{|c|c|}
\hline
Kursnummer & Dozent\\ \hline
a1 & m \\
... & ... \\
\hline
\end{tabular}

Kursplan: \\
\begin{tabular}{|c|c|}
\hline
Kursnummer & Voraussetzung\\ \hline
c2 & a0 \\
... & ... \\
\hline
\end{tabular}

Raumbelegung: \\
\begin{tabular}{|c|c|c|}
\hline
Kursnummer & Raum & Zeit\\ \hline
a1 & r1 & t1 \\
... & ... \\ & ... \\
\hline
\end{tabular}

Belegung: \\
\begin{tabular}{|c|c|c|}
\hline
Kursnummer & Teilnehmer & Wiederholer\\ \hline
a1 & s & j \\
... & ... \\ & ... \\
\hline
\end{tabular}

\begin{equation}
\begin{split}
vs(X, Y) &:- Kp(X,Y). \\
vs(X, Y) &:- vs(X, Z), Kp(Z, Y). \\
stdpl(W,X,Y,Z) &:- bl(X, W, V), rb(X,Y,Z). \\
ueberschneidungen(X,Y) &:- Kp(Z, X), Kp(Z, Y), rb(X, V,T), rb(Y, W, T), X \neq Y.
\end{split}
\end{equation}


\subsubsection*{Deklarative Semantik}
Extensionale Prädikate eines Programms (ext. Rel, ext. DB): EDB\\
Intentionale Prädikate eines Programms (int. Rel, int. DB): IDB\\

\paragraph*{Bedeutung}
Bedeutung eines Datalog-Programms P: Menge derjenigen Grundatome zu den intentionalen Prädikaten von P, die logisch aus P gefolgert werden können. (Jedes Modell von P ist auch ein Modell von $f \in F$). Mit Zielklausel g(...), $g \in Praed$: Aus P logisch folgerbare Grundatome zu g, die von g(..) subsummiert (überdeckt) werden.
\begin{tabular}{c}
g(a,X) \\ \hline
g(a,b) \\
g(a,c)
\end{tabular}

In P werden Werte aus Wertebereichen verwendet, ebenso in Darstellung der extensionalen Prädikate als DB-Relation. Daher können wir $Kost_A$ und Dom idefntifizieren. Mithilfer der Herbrand Interpretation kann die Semantik festgelegt werden (ist möglich).


\paragraph*{Herbrand-Interpretation /-Modelle}

Gewöhnliche Interpretation: \\
\begin{equation}
\begin{split}
Konst_A &= \{ a,b \}, Dom = \{ \circ, \square \} \\
k(a) &= \circ\\
k(b) &= \square \\
ext(p(.,.)) &=  \{ (\circ, \square), (\square, \square) 
\end{split}
\end{equation}

eine mögliche Herbrand-Interpretation (passt dazu)
\begin{equation}
\begin{split}
Konst_A &= Dom = \{ a,b \} \\
k(a) &= a\\
k(b) &= b \\
ext(p(.,.)) &=  \{ (a, b), (b, b) 
\end{split}
\end{equation}

Entsprechende Herbrand-Interpretation. Betrachte alle Paare zu p(.,.), teste gemäß gegebener (gewöhnlicher) Interpretationin ext(p(.,.)).
\begin{equation}
\begin{split}
Konst_A &= \{ a,b \}, Dom = \{ \circ, \square \} \\
k(a) &= \square\\
k(b) &= \square \\
ext(p(.,.)) &=  \{ (\circ, \square), (\square, \square) 
\end{split}
\end{equation}
(a,a) wird zu $(\square, \square) \in ext(p(.,.))$

Herbrand-Interpretation
\begin{equation}
\begin{split}
Konst_A &= Dom = \{ a,b \} \\
k(a) &= a\\
k(b) &= b \\
ext(p(.,.)) &=  \{ (a,a), (a, b), (b,a), (b,b) \}
\end{split}
\end{equation}

Bei beiden Interpretationen sind die gleichen Formeln gültig bei Beschränkung auf quantorenfreie Formeln ohne Variablen (und ohne Funktionen).

\paragraph{Beispiel}

Erste Interpretation: $p(a,b) \wedge p(b,a) \Rightarrow (\square, \square) \in ext(p(.,.)) \wedge ...$ bzw. $(a,b) \in ext(p(.,.)) \wedge (b,a) \in ext(p(.,.))$

Menge von Konstanten und Prädikatensymbole ist endlich, daher ist die Anzahl der möglichen Herbrand-Interpreationen endlich.

\subsubsection*{Satz von Gödel / Skolem (Herbrand, 1930)}
Eine Klauselmenge P hat ein Modell genau dann wenn P hat ein Herbrand-Modell.
Daraus folgt, dass ein Verfahren analog zu Wahrheitstabellen in der Aussagenlogik möglich ist.

\paragraph{Beispiel}
$F = \{ p(a) \Rightarrow q(b), p(a) \wedge q(b) \}$, $q(b)?$

\begin{tabular}{|c|c|c|c|c|}
\hline
p & q & $p(a) \Rightarrow q(b) \text{erfüllt?}$ & $p(a) \wedge q(b) \text{erfüllt?}$ & $p(a) \Rightarrow q(b) \text{und} p(a) \wedge q(b) \text{erfüllt?}$\\ \hline
$\{\}$ & $\{\}$ & \checkmark & - & - \\
$\{\}$ & $\{ a \}$ & \checkmark & - & - \\
$\{\}$ & $\{ b \}$ & \checkmark & \checkmark & \checkmark \\
$\{\}$ & $\{ a, b \}$ & \checkmark & \checkmark & \checkmark \\
$\{ a \}$ & $\{\}$ & - & \checkmark & - \\
... & ... & ... & ... & .... \\
$\{ b \}$ & $\{ b \}$ & \checkmark & \checkmark & \checkmark
\end{tabular}

Jedes Modell von F ist auch ein Modell von q(b), d.h. q(b) kann aus F logisch gefolgert werden. Gilt bei Klauselmengen, aber \textbf{Vorsicht bei allgemeinen Formeln}.
\paragraph{Beispiel:} $\{p(a), (\exists X) (\lnot p(X))\}$ Formelmenge, keine Klauselmenge

Modell (vgl. Übung): 
\begin{equation}
\begin{split}
Dom &= \{ 0, 1 \} \\
k(a) &= 0\\
ext(p(.)) &=  \{ (0) \} 
\end{split}
\end{equation}

\paragraph{Aber:} Es gibt kein durch \texttt{ext} bestimmtes Herbrand-Modell:

\begin{enumerate}
\item $ext(p(.)) = \{(a)\}, Konst = Dom = \{ a \}$
\item $ext(p(.) = \{\})$
\end{enumerate}

Herbrand-Modell muss genügend viele Elemente enthalten, damit der Satz von Gödel / Skolem gelten kann. \textbf{Skolemisierung} bedeutet, dass man alle Existenzquantoren durch Funktionen ersetzt:

\begin{equation}
(\forall x_1,....,x_n)(\exists y) (F) \leadsto (\forall x_1,...,x_n)(F[f(x_1,...x_n) / y])
\end{equation}


\subsubsection{Bemerkung: Skolemisierung}

Jede Formel der PL1 Logik kann man in einer \underline{erfüllbarkeitsäquivalente} Formel in Skolemform umformen:
\begin{enumerate}
\item Pränexnormalform
\item Umformungen à la $(\forall x_1,....,x_n)(\exists y) (F) \leadsto (\forall x_1,...,x_n)(F[f(x_1,...x_n) / y])$ mit jeweils einem neuen Funktionssymbol
\end{enumerate}
Dies ist eine Art ``Materialisierung'' der durch den Existenzquantor gebundenen Variablen.

\paragraph{Beispiel (von oben)}

$\{p(a), \lnot p(y)\footnote{neue Variable} \}$ erfüllbar $\Longleftrightarrow \{p(a), (\exists X)(\lnot p(X))\}$ erfüllbar \\

\textbf{Vorsicht:} Semantische Äquivalenz von Formeln und ihren Skolem-Normalformen im Allgemeinen bicht gegeben.

Skolem-NF: $(\exists X)(p(X)) : p(a)$


\end{document}